\chapter{Общая структура пояснительной записки}\label{app-structure}
%\addcontentsline{toc}{chapter}{}

\refsection

\begin{enumerate}
	\item Титульный лист %(в данном примере используется титульный лист от преддипломной практики)
	\item Лист с подписями (только для ВКР)
	\item Задание (в данном примере используется задание на диплом)
	\item Отчет из Антиплагиага \footnote{Обычно, допускается до 30\% заимствованного текста для работ бакалавров и до 20\% -- для работ магистров; см. соответствующие нормативные документы}
	\item Реферат (всегда на отдельной стр.)%, и эта страница \textit{НЕ} нумеруется)
	\item Оглавление. Начинается с новой страницы. %Обычно, это первая нумеруемая страница.
	\item Введение
	\begin{enumerate}
		\item Актуальность
		\item Новизна
		\item Оригинальная суть исследования
		\item Содержание ПЗ по главам (тезисно)
	\end{enumerate}
	\item Аналитическая глава. Пишется в стиле \textit{аналитического обзора}
	\item Теоретическая и инженерная глава. Описываются использованные, доработанные и разработанные модели, алгоритмы, методы, и т.п. Кроме того, тут формулируется архитектура системы.
	\item Инженерная глава. В этой главе следует отразить результаты проектирования, что, в общем случае, включает в себя следующие пункты:
	\begin{enumerate}
		\item Описание используемой методики проектирования
		\item Общая архитектура системы
		\item Архитектура подсистемы [таких подразделов может быть несколько штук, по одному на каждую подсистему или модуль, требующую детальное рассмотрение]
		\item Проектирование внешних и внутренних интерфейсов/протоколов взаимодействия
	\end{enumerate}
	\item Практическая глава. Описывается реализация, включая выбор инструментальных средств \footnote{В тех случаях, когда \begin{inparaenum}[(a)]\item этот выбор имеет существенное значение для всей работы и \item он не был, по каким-либо причинам, проделан в аналитической главе \end{inparaenum}}. Типовое содержание:
	\begin{enumerate}
		\item Состав и структура реализованного ПО 
		\item Выбор инструментальных средств
		\item Основные сценарии работы различных категорий пользователей
		\item Результаты тестирования (разработка тестовых примеров, таблицы и графики результатов прогона)
		\item Сравнение с существующими аналогами
	\end{enumerate}
	\item Заключение
	\item Список литературы 
	\item Приложения
\end{enumerate}

Кроме того, в ПЗ могут включаться и такие разделы, как словарь терминов, 
список сокращений и др. В зависимости от предпочтений автора, могут 
помещаться как в начале ПЗ (до оглавления), так и в конце (после заключения, 
но до приложений).

\paragraph{Замечания}: \mynobreakpar %\nopagebreak

\begin{enumerate}

  \item На каждый элемент из списка литературы должна быть хотя
бы одна ссылка в тексте.

  \item Список литературы должен быть оформлена согласно ГОСТ
\cite{Gost.7.1.2003,Gost.7.0.5.2008}.

  \item Минимальное количество источников для УИРов --- 15--20 (для
работ с большой аналитической и теоретической частью нормальное количество ---
25-30 и более), для дипломов --- соответственно, 30--35 и 35--60. Эти цифры
существенны, т.к. <<недобор>>, как правило, свидетельствует о не выполнении
аналитической части работы и, следовательно, недостаточном владении предметом.

  \item При подготовке РСПЗ рекомендуется вставлять уже наработанные к 
моменту подготовки РСПЗ материалы. Однако, в любом случае, каждый раздел 
должен начинаться с аннотации, заключенной в окружение \verb|\annotation|. В 
пояснительной записке к диплому аннотации не нужны. 

  \item Между заголовком главы и первым разделом рекомендуется поместить один-два абзаца связанного текста с кратким содержанием (планом) главы.

  \item Общее число и объем приложений не ограничивается. Объем ПЗ
\textbf{\textit{без}} приложений --- 25--40 стр. для УИРов, и не менее 60--100
стр. для дипломов. Объем ПЗ не может быть меньше указанных размеров. Это
означает, что студент не выполнил работу, или, как минимум, не удосужился
подготовить адекватную ПЗ. Превышать верхние пределы также не желательно, в
некоторых комиссиях это может восприниматься негативно; однако, в целом,
небольшое превышение допустимо, если проделана действительно большая работа и
получено много результатов (например, экспериментальных, или получены
нетривиальные аналитические или теоретические результаты).

  \item ГОСТ требует, чтобы нумерация страниц начиналась с 
первой, титульной, страницы. При этом на самой титульной странице номер не 
печатается. В данном случае, номера также не проставляются на листах задания, 
а также на листе с подписями (для ВКР).

\end{enumerate}

\printbibliography[heading=subbibliography]

\endrefsection

%%% Local Variables:
%%% TeX-engine: xetex
%%% eval: (setq-local TeX-master (concat "../" (seq-find (-cut string-match ".*-3-pz\.tex$" <>) (directory-files ".."))))
%%% End:
