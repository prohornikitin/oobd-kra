\chapter*{Заключение}
\addcontentsline{toc}{chapter}{Заключение}

В заключении в тезисной форме необходимо отразить результаты работы:

\begin{itemize}
	\item аналитические (что изучено/проанализировано);
	\item теоретические;
	\item инженерные (что спроектировано);
	\item практические (что реализовано/внедрено).
\end{itemize}

Примерная формула такая: по каждому указанному пункту приводится по 3-5 результатов, каждый результат излагается в объеме до 5 фраз или предложений.

Также есть смысл привести предполагаемые направления для будущей работы.

Общий объем заключения не должен превышать 1,5 страниц (1 страницы для УИРов).

%%% Local Variables:
%%% TeX-engine: xetex
%%% eval: (setq-local TeX-master (concat "../" (seq-find (-cut string-match ".*-3-pz\.tex$" <>) (directory-files ".."))))
%%% End:
