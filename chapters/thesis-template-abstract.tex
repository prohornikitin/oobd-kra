\chapter*{Реферат}
\thispagestyle{plain}

Общий объем основного текста, без учета приложений ---
\pageref{end_of_main_text}.
Количество использованных источников~--- ХХ.
Количество приложений~--- Х.

%Ключевые слова: 
% \noindent \uppercase{ключевое слово 1, ключевое слово 2, \dots .}

% Целью данной работы является \dots

В первой главе изложены теоретические основы обработки временных рядов, будет объяснено понятие микросервисной архитектуры, и обозначены задачи, в которых её использование в контексте обработки временных рядов уместно. 

Во второй главе будет описана реализация пресказания временного ряда как отдельного сервиса потоковой обрабоки данных. 

% В третьей главе приводится описание программной реализации и экспериментальной проверки \dots.

% В приложении \ref{app-format} описаны основные требования к форматированию пояснительных записок к дипломам и (магистерским) диссертациям.

% В приложении \ref{app-structure} представлена общая структура пояснительной записки.

% В приложении \ref{app-manual} приведены некоторые дополнительные комментарии к использованию данного шаблона.

%%% Local Variables:
%%% TeX-engine: xetex
%%% eval: (setq-local TeX-master (concat "../" (seq-find (-cut string-match ".*-3-pz\.tex$" <>) (directory-files ".."))))
%%% End:
